\documentclass[UTF8]{ctexart}
\usepackage{listings}
\usepackage{graphicx}
\author{孙嘉玺\\
\texttt{22920162204038}}
\date{\today}
\title{查重程序实验报告}

\begin{document}
\maketitle
\tableofcontents
\newpage

\section{题目描述}
给两个程序文件,判断程序是否存在抄袭,输出两个程序的相似度,根据相似都来判读是否存是抄袭。
\section{算法实现}
\subsection{算法思想}
查重算法主要借鉴最长公共字串的思想,不过需要调整一些东西。
\\第一就是文本的最小单位不应该是字符,而是一个单词。因为在抄袭的时候,是在抄袭关键字的级别上发生的。\\
另一个需要调整的就是要把文本中无效的字符除去防止干扰查重,其中一些字符就是换行符和空格,tab键等,最终将文本删去无效字符,只留下关键字。
\subsection{数据结构}
存储文本的数据结构在python中表现为一个list,list中的元素是单词。
存储动态规划结果的是一个numpy\ array\ 来存储动态规划表。
\subsection{程序运行环境}
\begin{enumerate}
\item python
\item terminal 
\item terminal 中使用命令 python FindCopy.py filename1 filename2 
\end{enumerate}
\subsection{程序}
本程序使用python编写
\lstset{language=python,
	frame=shadowbox}
\begin{lstlisting}
# coding: utf-8

from __future__ import division, print_function
from io import open
import numpy as np
import sys


def get_list(filename1, filename2):
    with open(filename1) as f:
        l = f.read()
        l = l.split()
        list1 = l
    with open(filename2) as f:
        l = f.read()
        l = l.split()
        list2 = l
    return list1, list2


def find_max_commen(seqs):
    dp = np.zeros(
        (len(seqs[0]), len(seqs[1])) 
    )
    
    for i in range(0, dp.shape[0]):
        if seqs[0][i] == seqs[1][0]:
            dp[i][0] = 1
    
    for j in range(0, dp.shape[1]):
        if seqs[1][j] == seqs[0][0]:
            dp[0][j] = 1
    
    for i in range(1, dp.shape[0]):
        for j in range(1, dp.shape[1]):
            if seqs[0][i] == seqs[1][j]:
                dp[i][j] = max(dp[i-1][j-1]+1, 
				dp[i][j-1], 
				dp[i-1][j], 
				dp[i][j])
    
    return max(dp.reshape(-1))



if __name__ == "__main__":
    filename1 = sys.argv[1]
    filename2 = sys.argv[2]
    file_pair = get_list(filename1, filename2)
    commen_num = find_max_commen(file_pair)
    print("%s %s 相似度为: %.2f " 
    		%(filename1, filename2 ,(
        	100 * commen_num / min(
        	len(file_pair[0]), len(file_pair[1]))
         	)), end='%\n')


\end{lstlisting}
\section{实验数据\&结果}

\subsection{实验数据}
实验数据有两个c++程序文件file1.cpp \ file2.cpp
\\
文件内容如下

\begin{itemize} 
\item file1.cpp
\lstset{language=c++,
	frame=shadowbox}
\begin{lstlisting}
#include <stdio.h>
#include <iostream>
#include <string>
#include <vector>

using namespace std;


int main() {
	printf("hello world");
	return 0;
}

\end{lstlisting}

\item file2.cpp
\lstset{language=c++,
frame=shadowbox}
\begin{lstlisting} 
#include <iostream>
#include <string>
#include <algorithm>
#include <vector>

using namespace std;

int main() {
	cout << "hello world" << endl;
	return ;
}

\end{lstlisting}
\end{itemize}


\subsection{实验结果}
程序输出结果如下
\begin{lstlisting}
file1.cpp file2.cpp 相似度为: 42.11 %
\end{lstlisting}

\section{实验总结}

\subsection{结果分析}
从分析结果来看,发现程序的相似度是没有问题的,然而难点是如何设定门限来确定是否存在抄袭,这是程序难以解决的,如果真正的要判读出是否抄袭,仅仅判断相似度是不够的,可能还要需要人工校对。另一方面,程序规则是死的,如果人在了解程序逻辑之后很可能会对文本做一些手脚,来欺骗程序,这也是这个程序不足的地方。
\subsection{总结}
作为一个算法实验,得到这样的实验结果还可以接受,但是要放到实践中,使这个程序能够用起来,可能需要深入打磨算法,可能还需要一些机器学习的算法,学无止境,继续努力。
\includegraphics[width=100pt, height=400pt]{}
\end{document}

